\documentclass[12pt]{article}

\author{Daniel Wysocki and Craig Graci}
\title{Modeling music using hidden Markov models}
\date{May 13, 2015}


\usepackage[backend=bibtex, style=authoryear]{biblatex}
\usepackage{amsmath,amsfonts,amssymb,mathrsfs,physics}
\PassOptionsToPackage{hyphens}{url}
\usepackage[hidelinks,colorlinks,urlcolor=blue,citecolor=blue,linkcolor=blue]{hyperref}
\usepackage{listings}

\DeclareMathOperator{\signature}{\mathcal{S}}

\addbibresource{semester-paper.bib}

\begin{document}

\maketitle

\begin{abstract}

  We model songs as a sequence of notes emitted by a hidden Markov model.
  A model is trained on a given song, and can be used to randomly generate new,
  similar songs, as well as study the structure of the composition.
  We produce a ``signature'' for each song, defined by two feature vectors,
  based on the emission ($\vb{e}$) and transition ($\vb{t}$) probabilities.

\end{abstract}

\section{Introduction}

Algorithmic music composition is a subject of much interest in the artificial
intelligence research community. Many approaches have been taken to produce
acceptable music, as the task is not straightforward. The classic work,
Experiments in Music Intelligence by \textcite{cope1996}, explores a range of
approaches, from knowledge-based systems, to machine learning approaches.
A knowledge-based system works by following a set of rules defined by the
programmer, and is very difficult to get right, due to its rigidity. Machine
learning, on the other hand, creates a model from existing compositions, which
are used as generators for new compositions. The latter approach is also
employed by \textcite{marchini2011}. In order to produce satisfying music, it
is often necessary to combine the two approaches in some manner.

The current work employs a pure machine learning approach, training a hidden
Markov model \parencite{rabiner1989} on a song (Section \ref{sec:hmm}). This
model is used as a generator (Section \ref{sec:alg-comp}), as well as a means
of observing the structure of the training songs
(Section \ref{sec:signatures}).


\section{Data}

\subsection{Format}

Training data were provided to the model using the JFugue \parencite{jfugue4}
MusicString format. The format is ideal, as it allows a song to be represented
as a simple linear sequence of notes. JFugue also has the ability to convert
MusicStrings into MIDI, for easy playback.


\subsection{Selected Compositions}

The training songs used were limited to simple melodies, listed in Table
\ref{tab:melodies}. The selected songs were taken from \textcite{durey2001},
who used hidden Markov models to identify melodies. While more complex songs
may be used in the future, such simple melodies may be more accurately modeled
by low-dimensional hidden Markov models.

\begin{table}[t]
\centering
\begin{tabular}{l|l}
& Title
\\\hline\hline
1 & Auld Lang Syne
\\
2 & Barbara Allen
\\
3 & Frere Jacques
\\
4 & Happy Birthday
\\
5 & I'm a Little Teapot
\\
6 & Mary Had a Little Lamb
\\
7 & Scarborough Fair
\\
8 & This Old Man
\\
9 & Three Blind Mice
\\
10 & Twinkle Twinkle Little Star
\end{tabular}
\caption{Table of melodies used, see \ref{app:melodies} for transcriptions.}
\label{tab:melodies}
\end{table}


\section{Modeling}

\subsection{Hidden Markov Model}
\label{sec:hmm}

A hidden Markov model is used to describe each song. A hidden Markov model can
be described by the following \parencite{rabiner1989}:

1) The set of hidden states, denoted
$S = \{ s_1, s_2, \ldots, s_N \}$.
At time $t$, the current state is denoted $q_t$. These states represent states
of the song, which are determined automatically, and are described in more
detail in Section \ref{sec:signatures}.

2) The set of possible observation symbols, denoted
$V = \{ v_1, v_2, \ldots, v_M \}$,
which are emitted by each state with a certain probability distribution. These
observations are the individual notes of the song. There is also a special
``end of song'' note appended to the song, which denotes the termination
of the song.

3) The state transition probability distribution
$A = \{ a_{ij} \}$,
where
\begin{equation}
  a_{ij} = P[ q_{t+1} = s_j | q_t = s_i ], \quad 1 \leq i, j \leq N.
\end{equation}
This defines the probability that the song will make a transition between each
pair of states.

4) The observation probability distribution in state $j$,
$B = \{ b_j(k) \}$,
where
\begin{align}
  b_j(k) = P[ v_k \text{ at } t | q_t = s_j ] \quad
& 1 \leq j \leq N \notag
\\
& 1 \leq k \leq M.
\end{align}
This defines the probability of a particular note being played while the song
is in a particular state.

5) The initial state distribution
$\pi = \{ \pi_i \}$, where
\begin{equation}
  \pi_i = P[ q_1 = s_i ], \quad 1 \leq i \leq N.
\end{equation}
This defines the probability of the song beginning in any given state.

A model is denoted concisely as $\lambda = (A, B, \pi)$, where $N$ and $M$ are
contained within $A$ and $B$. Once a model is created -- either with random or
uniform probabilities -- it is improved to better describe a sequence of
observations, $O = (o_1, o_2, o_3, \ldots, o_T)$. This is done using the
Baum--Welch algorithm \parencite{baum1966}, which iteratively improves the
model in order to find a local maxima for the likelihood $P[O|\lambda]$.
Likelihood is determined using the forward algorithm. These algorithms were
implemented according to \textcite{mann2006} and \textcite{ibe2013}.


\subsection{Model Selection}

When the system being modeled has well defined states, $N$ can be chosen such
that each state is described uniquely by the model. However, in the case of a
song, it is not entirely clear what a state corresponds to. Therefore, it was
necessary to implement a simple algorithm for selecting $N$.

A minimum and maximum number of states, $N_{\rm min}$ and $N_{\rm max}$ are
provided manually, in order to restrict the search space. Then the search space
is divided into a number of bins, $b$. A model is trained at $N_{\rm max}$, and
$b$ (or fewer) equally spaced integer values of $N_i$, such that
$N_{\rm min} \leq N < N_{\rm max}$. The likelihood, $P[O|\lambda]$, is computed
for each model, and the model which maximizes that likelihood is taken. If
multiple models maximize the likelihood, the one with the lowest $N$ is
taken. The procedure then repeats with $N_{\rm min} = N_{i-1}$, and
$N_{\rm max} = N_i$, until the search space cannot be further subdivided.


\subsection{Algorithmic Composition} \label{sec:alg-comp}

Once a model has been obtained for a song, it can be used as a generator for
new songs. The procedure for song generation is as follows:

\begin{enumerate}
\item With probability $\pi_i$, let the initial state $q_1 = s_i$.

\item When at time $t$, the current state is $q_t = s_j$, emit note $v_k$ with
  probability $b_j(k)$. If that note is the ``end of song'' note, terminate,
  otherwise let $q_{t+1} = s_i$ with probability $a_{ji}$ and repeat this step.
\end{enumerate}

Once the procedure terminates, the notes are converted to a MIDI file using
JFugue. Ten compositions have been made for each of the songs of interest, and
can be listened to at

\url{https://dwysocki.github.io/csc466/simple_compositions/}




\subsection{Signatures} \label{sec:signatures}

A ``signature'' is computed for each song, based on its model. It is a 2-tuple
$\signature = (\vb{e}, \vb{t})$, where $\vb{e}$ is the emission feature
vector, and $\vb{t}$ is the transition feature vector.

$\vb{e}$ is defined as
\begin{equation}
  \vb{e} = \langle e_1, e_2, e_3, \ldots \rangle,
\end{equation}
where $e_i$ is the number of states in the model with $i$ notes likely to be
emitted. Here, ``likely'' means that the probability of a note being emitted is
more than one standard deviation above the mean for that state. If $e_1$ is the
dominant term, that means that the majority of states correspond to precisely
one note. Interestingly, $e_2 = e_3 = \ldots = 0$ for every song except
Barbara Allen, which has $e_2 = 1$.

$\vb{t}$ is defined as
\begin{equation}
  \vb{t} = \langle t_1, t_2, t_3, \ldots \rangle,
\end{equation}
where $t_i$ is the number of states in the model with $i$ states likely to be
transitioned to. When a state only transitions to one state, that state
is part of a ``linear progression'', and if it transitions to $n > 1$ states,
it is an ``$n$-way branch''. Linear progressions comprised the majority of
states for all songs tested, with varying distributions of $n$-way branches.
Table \ref{tab:signatures} lists the signatures of the songs of interest.

\begin{table}[t]
\centering
\begin{tabular}{l||l|llllll|ll}
  \textbf{Title}
& $N$ & $t_1$ & $t_2$ & $t_3$ & $t_4$ & $t_5$ & $t_6$ & $e_1$ & $e_2$
\\\hline\hline
  Auld Lang Syne
& 55
& 32 & 18 & 5 & 0 & 0 & 0
& 55 & 0
\\
  Barbara Allen
& 55
& 31 & 23 & 1 & 0 & 0 & 0
& 54 & 1
\\
  Frere Jacques
& 60
& 33 & 17 & 8 & 1 & 1 & 0
& 60 & 0
\\
  Happy Birthday
& 54
& 34 & 17 & 3 & 0 & 0 & 0
& 54 & 0
\\
  I'm a Little Teapot
& 43
& 32 & 9 & 2 & 0 & 0 & 0
& 43 & 0
\\
  Mary Had a Little Lamb
& 47
& 25 & 11 & 11 & 0 & 0 & 0
& 47 & 0
\\
  Scarborough Fair
& 53
& 44 & 7 & 2 & 0 & 0 & 0
& 53 & 0
\\
  This Old Man
& 49
& 24 & 17 & 7 & 1 & 0 & 0
& 49 & 0
\\
  Three Blind Mice
& 60
& 25 & 12 & 15 & 5 & 2 & 1
& 60 & 0
\\
  Twinkle Twinkle Little Star
& 33
& 15 & 11 & 6 & 2 & 0 & 0
& 34 & 0
\end{tabular}
\caption{Signatures for the selected songs.}
\label{tab:signatures}
\end{table}


\section{Conclusion}

This work has demonstrated that hidden Markov models are capable of capturing
some of the essence of music, and serve as effective generators. However, to
create truly satisfactory music, one will have to extend the method beyond a
simple hidden Markov model. Extensions of the hidden Markov model, such as the
left--right HMM \parencite{rabiner1989} and the 
generalized HMM \parencite{kulp1996} may be able to perform better.

Further investigation into the song models should be performed. The feature
vectors may be used for $k$-means clustering. Linear progressions may also be
extracted, identifying the defining parts of songs. 



\printbibliography


\appendix

\section{Song Transcriptions}
\label{app:melodies}

The JFugue MusicStrings used as training data are provided here verbatim.

\textbf{Auld Lang Syne}
\lstinputlisting{songs/auld-lang-syne.jfugue}

\textbf{Barbara Allen}
\lstinputlisting{songs/barbara-allen.jfugue}

\textbf{Frere Jacques}
\lstinputlisting{songs/frere-jacques.jfugue}

\textbf{Happy Birthday}
\lstinputlisting{songs/happy-birthday.jfugue}

\textbf{I'm a Little Teapot}
\lstinputlisting{songs/im-a-little-teapot.jfugue}

\textbf{Mary Had a Little Lamb}
\lstinputlisting{songs/mary-had-a-little-lamb.jfugue}

\textbf{Scarborough Fair}
\lstinputlisting{songs/scarborough-fair.jfugue}

\textbf{This Old Man}
\lstinputlisting{songs/this-old-man.jfugue}

\textbf{Three Blind Mice}
\lstinputlisting{songs/three-blind-mice.jfugue}

\textbf{Twinkle Twinkle Little Star}
\lstinputlisting{songs/twinkle-twinkle-little-star.jfugue}



\end{document}
